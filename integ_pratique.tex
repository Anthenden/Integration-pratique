\newif\ifwhole

\wholetrue
% Ajouter \wholetrue si on compile seulement ce fichier

\ifwhole
 \documentclass[a4page,10pt]{article}
     \usepackage[Latin1]{inputenc}
\usepackage[francais]{babel}
\usepackage{amsmath,amssymb,amsthm}
\usepackage{textcomp}
\usepackage{mathrsfs}
\usepackage{algcompatible,algorithm  }
\usepackage[all]{xy}
\usepackage{hyperref}
\usepackage{fancyhdr}
\usepackage{supertabular}
\pagestyle{plain}
\makeatletter

\renewcommand*{\ALG@name}{Algorithme}

\makeatother

\renewcommand{\algorithmicrequire}{\textbf{\textsc {Entrées  :  } } }
\renewcommand{\algorithmicensure}{\textbf{\textsc { Sortie  :  } } }
\renewcommand{\algorithmicwhile}{\textbf{Tant que}}
\renewcommand{\algorithmicdo}{\textbf{faire }}
\renewcommand{\algorithmicif}{\textbf{Si}}
\renewcommand{\algorithmicelse}{\textbf{Sinon}}
\renewcommand{\algorithmicthen}{\textbf{alors }}
\renewcommand{\algorithmicend}{\textbf{fin}}
\renewcommand{\algorithmicfor}{\textbf{Pour}}
\renewcommand{\algorithmicuntil}{\textbf{Jusqu'à}}
\renewcommand{\algorithmicrepeat}{\textbf{Répéter}}

\newcommand{\A}{\mathbb{A}}
\newcommand{\B}{\mathbb{B}}
\newcommand{\C}{\mathbb{C}}
\newcommand{\D}{\mathbb{D}}
\newcommand{\E}{\mathbb{E}}
\newcommand{\F}{\mathbb{F}}
\newcommand{\G}{\mathbb{G}}
\renewcommand{\H}{\mathbb{H}}
\newcommand{\I}{\mathbb{I}}
\newcommand{\J}{\mathbb{J}}
\newcommand{\K}{\mathbb{K}}
\renewcommand{\L}{\mathbb{L}}
\newcommand{\M}{\mathbb{M}}
\newcommand{\N}{\mathbb{N}}
\renewcommand{\O}{\mathbb{O}}
\renewcommand{\P}{\mathbb{P}}
\newcommand{\Q}{\mathbb{Q}}
\newcommand{\R}{\mathbb{R}}
\renewcommand{\S}{\mathbb{S}}
\newcommand{\T}{\mathbb{T}}
\newcommand{\U}{\mathbb{U}}
\newcommand{\V}{\mathbb{V}}
\newcommand{\W}{\mathbb{W}}
\newcommand{\X}{\mathbb{X}}
\newcommand{\Y}{\mathbb{Y}}
\newcommand{\Z}{\mathbb{Z}}
\newcommand{\Acal}{\mathcal{A}}
\newcommand{\Bcal}{\mathcal{B}}
\newcommand{\Ccal}{\mathcal{C}}
\newcommand{\Dcal}{\mathcal{D}}
\newcommand{\Ecal}{\mathcal{E}}
\newcommand{\Fcal}{\mathcal{F}}
\newcommand{\Gcal}{\mathcal{G}}
\newcommand{\Hcal}{\mathcal{H}}
\newcommand{\Ical}{\mathcal{I}}
\newcommand{\Jcal}{\mathcal{J}}
\newcommand{\Kcal}{\mathcal{K}}
\newcommand{\Lcal}{\mathcal{L}}
\newcommand{\Mcal}{\mathcal{M}}
\newcommand{\Ncal}{\mathcal{N}}
\newcommand{\Ocal}{\mathcal{O}}
\newcommand{\Pcal}{\mathcal{P}}
\newcommand{\Qcal}{\mathcal{Q}}
\newcommand{\Rcal}{\mathcal{R}}
\newcommand{\Scal}{\mathcal{S}}
\newcommand{\Tcal}{\mathcal{T}}
\newcommand{\Ucal}{\mathcal{U}}
\newcommand{\Vcal}{\mathcal{V}}
\newcommand{\Wcal}{\mathcal{W}}
\newcommand{\Xcal}{\mathcal{X}}
\newcommand{\Ycal}{\mathcal{Y}}
\newcommand{\Zcal}{\mathcal{Z}}
\newcommand{\Ascr}{\mathscr{A}}
\newcommand{\Bscr}{\mathscr{B}}
\newcommand{\Cscr}{\mathscr{C}}
\newcommand{\Dscr}{\mathscr{D}}
\newcommand{\Escr}{\mathscr{E}}
\newcommand{\Fscr}{\mathscr{F}}
\newcommand{\Gscr}{\mathscr{G}}
\newcommand{\Hscr}{\mathscr{H}}
\newcommand{\Iscr}{\mathscr{I}}
\newcommand{\Jscr}{\mathscr{J}}
\newcommand{\Kscr}{\mathscr{K}}
\newcommand{\Lscr}{\mathscr{L}}
\newcommand{\Mscr}{\mathscr{M}}
\newcommand{\Nscr}{\mathscr{N}}
\newcommand{\Oscr}{\mathscr{O}}
\newcommand{\Pscr}{\mathscr{P}}
\newcommand{\Qscr}{\mathscr{Q}}
\newcommand{\Rscr}{\mathscr{R}}
\newcommand{\Sscr}{\mathscr{S}}
\newcommand{\Tscr}{\mathscr{T}}
\newcommand{\Uscr}{\mathscr{U}}
\newcommand{\Vscr}{\mathscr{V}}
\newcommand{\Wscr}{\mathscr{W}}
\newcommand{\Xscr}{\mathscr{X}}
\newcommand{\Yscr}{\mathscr{Y}}
\newcommand{\Zscr}{\mathscr{Z}}
\newcommand{\Afrak}{\mathfrak{A}}
\newcommand{\Bfrak}{\mathfrak{B}}
\newcommand{\Cfrak}{\mathfrak{C}}
\newcommand{\Dfrak}{\mathfrak{D}}
\newcommand{\Efrak}{\mathfrak{E}}
\newcommand{\Ffrak}{\mathfrak{F}}
\newcommand{\Gfrak}{\mathfrak{G}}
\newcommand{\Hfrak}{\mathfrak{H}}
\newcommand{\Ifrak}{\mathfrak{I}}
\newcommand{\Jfrak}{\mathfrak{J}}
\newcommand{\Kfrak}{\mathfrak{K}}
\newcommand{\Lfrak}{\mathfrak{L}}
\newcommand{\Mfrak}{\mathfrak{M}}
\newcommand{\Nfrak}{\mathfrak{N}}
\newcommand{\Ofrak}{\mathfrak{O}}
\newcommand{\Pfrak}{\mathfrak{P}}
\newcommand{\Qfrak}{\mathfrak{Q}}
\newcommand{\Rfrak}{\mathfrak{R}}
\newcommand{\Sfrak}{\mathfrak{S}}
\newcommand{\Tfrak}{\mathfrak{T}}
\newcommand{\Ufrak}{\mathfrak{U}}
\newcommand{\Vfrak}{\mathfrak{V}}
\newcommand{\Wfrak}{\mathfrak{W}}
\newcommand{\Xfrak}{\mathfrak{X}}
\newcommand{\Yfrak}{\mathfrak{Y}}
\newcommand{\Zfrak}{\mathfrak{Z}}
\newcommand{\afrak}{\mathfrak{a}}
\newcommand{\bfrak}{\mathfrak{b}}
\newcommand{\cfrak}{\mathfrak{c}}
\newcommand{\dfrak}{\mathfrak{d}}
\newcommand{\efrak}{\mathfrak{e}}
\newcommand{\ffrak}{\mathfrak{f}}
\newcommand{\gfrak}{\mathfrak{g}}
\newcommand{\hfrak}{\mathfrak{h}}
\newcommand{\ifrak}{\mathfrak{i}}
\newcommand{\jfrak}{\mathfrak{j}}
\newcommand{\kfrak}{\mathfrak{k}}
\newcommand{\lfrak}{\mathfrak{l}}
\newcommand{\mfrak}{\mathfrak{m}}
\newcommand{\nfrak}{\mathfrak{n}}
\newcommand{\ofrak}{\mathfrak{o}}
\newcommand{\pfrak}{\mathfrak{p}}
\newcommand{\qfrak}{\mathfrak{q}}
\newcommand{\rfrak}{\mathfrak{r}}
\newcommand{\sfrak}{\mathfrak{s}}
\newcommand{\tfrak}{\mathfrak{t}}
\newcommand{\ufrak}{\mathfrak{u}}
\newcommand{\vfrak}{\mathfrak{v}}
\newcommand{\wfrak}{\mathfrak{w}}
\newcommand{\xfrak}{\mathfrak{x}}
\newcommand{\yfrak}{\mathfrak{y}}
\newcommand{\zfrak}{\mathfrak{z}}
\newcommand{\Abar}{\overline{A}}
\newcommand{\Bbar}{\overline{B}}
\newcommand{\Cbar}{\overline{C}}
\newcommand{\Dbar}{\overline{D}}
\newcommand{\Ebar}{\overline{E}}
\newcommand{\Fbar}{\overline{F}}
\newcommand{\Gbar}{\overline{G}}
\newcommand{\Hbar}{\overline{H}}
\newcommand{\Ibar}{\overline{I}}
\newcommand{\Jbar}{\overline{J}}
\newcommand{\Kbar}{\overline{K}}
\newcommand{\Lbar}{\overline{L}}
\newcommand{\Mbar}{\overline{M}}
\newcommand{\Nbar}{\overline{N}}
\newcommand{\Obar}{\overline{O}}
\newcommand{\Pbar}{\overline{P}}
\newcommand{\Qbar}{\overline{Q}}
\newcommand{\Rbar}{\overline{R}}
\newcommand{\Sbar}{\overline{S}}
\newcommand{\Tbar}{\overline{T}}
\newcommand{\Ubar}{\overline{U}}
\newcommand{\Vbar}{\overline{V}}
\newcommand{\Wbar}{\overline{W}}
\newcommand{\Xbar}{\overline{X}}
\newcommand{\Ybar}{\overline{Y}}
\newcommand{\Zbar}{\overline{Z}}
\newcommand{\abar}{\overline{a}}
\newcommand{\bbar}{\overline{b}}
\newcommand{\cbar}{\overline{c}}
\newcommand{\dbar}{\overline{d}}
\newcommand{\ebar}{\overline{e}}
\newcommand{\fbar}{\overline{f}}
\newcommand{\gbar}{\overline{g}}
\renewcommand{\hbar}{\overline{h}}
\newcommand{\ibar}{\overline{i}}
\newcommand{\jbar}{\overline{j}}
\newcommand{\kbar}{\overline{k}}
\newcommand{\lbar}{\overline{l}}
\newcommand{\mbar}{\overline{m}}
\newcommand{\nbar}{\overline{n}}
\newcommand{\obar}{\overline{o}}
\newcommand{\pbar}{\overline{p}}
\newcommand{\qbar}{\overline{q}}
\newcommand{\rbar}{\overline{r}}
\newcommand{\sbar}{\overline{s}}
\newcommand{\tbar}{\overline{t}}
\newcommand{\ubar}{\overline{u}}
\newcommand{\vbar}{\overline{v}}
\newcommand{\wbar}{\overline{w}}
\newcommand{\xbar}{\overline{x}}
\newcommand{\ybar}{\overline{y}}
\newcommand{\zbar}{\overline{z}}
\newcommand\bigzero{\makebox(0,0){\text{\huge0}}}
\newcommand{\limp}{\lim\limits_{\leftarrow}}
\newcommand{\limi}{\lim\limits_{\rightarrow}}


\DeclareMathOperator{\End}{\mathrm{End}}
\DeclareMathOperator{\Hom}{\mathrm{Hom}}
\DeclareMathOperator{\Vect}{\mathrm{Vect}}
\DeclareMathOperator{\multideg}{\mathrm{multideg}}
\DeclareMathOperator{\LM}{\mathrm{LM}}
\DeclareMathOperator{\LT}{\mathrm{LT}}
\DeclareMathOperator{\LC}{\mathrm{LC}}
\DeclareMathOperator{\PPCM}{\mathrm{PPCM}}
\DeclareMathOperator{\PGCD}{\mathrm{PGCD}}
\DeclareMathOperator{\Syl}{\mathrm{Syl}}
\DeclareMathOperator{\Res}{\mathrm{Res}}
\DeclareMathOperator{\Com}{\mathrm{Com}}



\headheight=0mm
\topmargin=-20mm
\oddsidemargin=-1cm
\evensidemargin=-1cm
\textwidth=18cm
\textheight=25cm
\parindent=0mm
\newif\ifproof
\newcommand{\demo}[1]{\ifproof #1 \else \fi}
 %Instruction d'utilisation : 
%les preuves du texte sont, en principe, entre des balises \demo, en sus des \begin{proof} pour l'instant.
%Laisser le texte tel quel, fait qu'elles ne sont pas affich�es.
%Mettre \prooftrue fait que toutes les preuves jusqu'� un \prooffalse ou la fin du document. 


 \begin{document}
\newtheorem{Thm}{Th�or�me}[section]
\newtheorem{prop}[Thm]{Proposition}
\newtheorem{propte}[Thm]{Propri�t�}
\newtheorem{Lemme}[Thm]{Lemme}
\newtheorem{cor}[Thm]{Corollaire}

\theoremstyle{definition}

\newtheorem{Ex}[Thm]{Exemple}
\newtheorem{Def}[Thm]{D�finition}
\newtheorem{Not}[Thm]{Notation}
\newtheorem{Conv}[Thm]{Convention}

\theoremstyle{remark}
\newtheorem{Rq}[Thm]{Remarque}
\newtheorem{Slog}[Thm]{Slogan}
\fi
\begin{Thm}[Convergence domin�e]
Soit $(f_n)$ une suite de fonctions de $L^1 $ qui converge $\mu $-presque partout vers une application mesurable $f $ de $ $ dans $\overline{\R} $ et tel qu'il existe une fonction $g \in L^1$ telle que $\forall n \in \N, \forall x \in X, |f_n(x)| \leq g(x)$. \\
Alors $f \in \L^1 $ et $\lim \|f_n-f \|=0$ i.e. $f_n$ converge vers $f$ dans $L^1$.
\end{Thm}
Dans les hypoth�ses de ce th�or�me, $\lim \int f_n=\int \lim f_n=\int f$
\vspace{4cm}
\begin{Thm}[de Fubini]
Soient $(X,\Fcal,\lambda) $ et $(Y,\Gcal,\mu)$ deux espaces mesur�s  $\sigma$-finis. Soit $f : X \times Y \to \overline{R}$ une application $\Fcal \otimes \Gcal$ -mesurable. \\
1) Si $f \geq 0$ alors les applications $\varphi_y : x \mapsto \int_Y f(x,y) d\mu(y)$ et $\psi_x : y \mapsto \int_x f(x,y) d\lambda(y)$ sont mesurables sur,respectivement, $\Fcal$ et $\Gcal$ et $$\int_{X \times Y} f d(\lambda \otimes \mu)=\int_{X} \varphi_y(x) d\lambda(x)=\int_{Y} \psi_x(y) d\mu(y) $$
2) Si $f$ est $\lambda \times \mu$-int�grable alors les applications $f(x,\cdot)$ et $f(\cdot,y)$  sont int�grables pour, respectivement, presque tout $x \in X$ et presque tout $y \in Y$. 
Les applications $\varphi_y$ et $\psi_x$ sont bien d�finis (comme dans le 1)) pour, respectivement, presque tout $y$ et presque $x$ et sont int�grables.
On a, alors, comme dans le cas pr�c�dent, $$\int_{X \times Y} f d(\lambda \otimes \mu)=\int_{X} \varphi_y(x) d\lambda(x)=\int_{Y} \psi_x(y) d\mu(y) $$





\end{Thm}
\vspace{4cm}
\section{Th�or�me de continuit� et de d�rivation sous le signe $\int$}

 Dans la suite, $a$ et $b $ sont deux r�els tels que $a <b $ et $f $ d�signe une application de $
X \times [a,b] $ dans $\R $.
%Pour tout $(_0,t_0)\in X \times   [a,b]$, on note $f (,\cdot) $ et $f (\cdot,t_0) $ les applications respectives $t \in [a,b] \mapsto f (_,t) \in \R $ et $ \in  \mapsto f(,t_0) \in \R $
\begin{Thm}[ de continuit� sous le signe $\int$ ]
Si $f $ v�rifie les trois hypoth�ses suivantes : \\
1) Pour tout $t \in [a,b] $, $f (\cdot,t) \in L^1$ ; \\
2) $\exists g \in L^1, \forall x \in X, \forall t \in [a,b],  |f(x,t)| \leq g(x)$ (hypoth�se de domination) \\
3) Pour tout $ \in X $, l'application  $f (,\cdot) $ est continue sur $[a,b] $ ;\\
alors, l'application  $F: t \in [a,b] \mapsto F(t):=\int_X f(x,t) d\mu(x) \in \R $ est continue sur $[a,b] $.\\
\end{Thm}
\begin{Thm}[de d�rivabilit� sous le signe $\int$ ]
Si $f $ v�rifie les trois hypoth�ses suivantes : \\
1) Pour tout $t \in [a,b] $, $f (\cdot,t) \in L^1$ ; \\
2) Pour tout $(,t)\in X \times [a,b]  $, l'application  $\frac{\partial f}{\partial t}(x,t) $ est d�finie ;\\
3) $\exists g \in L^1, \forall x \in X, \forall t \in [a,b],  \left|\frac{\partial f}{\partial t}(x,t)\right| \leq g(x)$ (hypoth�se de domination). \\
Alors,\\
 1)l'application  $F: t \in [a,b] \mapsto F(t):=\int_X f(x,t) d\mu(x) \in \R $ est d�rivable sur $[a,b] $\\
2) Pour tout $t \in ]a,b [$, $\frac{\partial f}{\partial t}(\cdot,t)\in L^1$ et $F'(t)=\int_X \frac{\partial f}{\partial t}(x,t) d\mu (x) $.
\end{Thm}
\begin{Thm}[G�n�ralisation ]
Soit $k \in \N$ et $f $ qui v�rifie les trois hypoth�ses suivantes : \\
1) Pour tout $t \in [a,b] $, $f (\cdot,t) \in L^1$ ; \\
2) Pour tout $(x,t)\in X \times [a,b]  $, l'application  $\frac{\partial^k f}{\partial t^k}(x,t) $ est d�finie ;\\
3) $\exists g \in L^1, \forall x \in X, \forall t \in [a,b],  \left|\frac{\partial^k f}{\partial t^k}(x,t)\right| \leq g(x)$ (hypoth�se de domination). \\
Alors,\\
 1)l'application  $F: t \in [a,b] \mapsto F(t):=\int_X f(x,t) d\mu(x) \in \R $ est $k$ fois d�rivable sur $[a,b] $\\
2) Pour tout $t \in ]a,b [$, $\frac{\partial^k f}{\partial t^k}(\cdot,t)\in L^1$ et $F^{(k)}(t)=\int_X \frac{\partial^k f}{\partial t^k}(x,t) d\mu (x) $.
\end{Thm}








\ifwhole
 \end{document}
\fi