\newif\ifwhole

\wholetrue
% Ajouter \wholetrue si on compile seulement ce fichier

\ifwhole
 \documentclass[a4page,10pt]{article}
     \input{entete}
 \begin{document}
\input{enonce}
\fi
\begin{Thm}[Convergence domin�e]
Soit $(f_n)$ une suite de fonctions de $L^1 $ qui converge $\mu $-presque partout vers une application mesurable $f $ de $ $ dans $\overline{\R} $ et tel qu'il existe une fonction $g \in L^1$ telle que $\forall n \in \N, \forall x \in X, |f_n(x)| \leq g(x)$. \\
Alors $f \in \L^1 $ et $\lim \|f_n-f \|=0$ i.e. $f_n$ converge vers $f$ dans $L^1$.
\end{Thm}
Dans les hypoth�ses de ce th�or�me, $\lim \int f_n=\int \lim f_n=\int f$
\vspace{4cm}
\begin{Thm}[de Fubini]
Soient $(X,\Fcal,\lambda) $ et $(Y,\Gcal,\mu)$ deux espaces mesur�s  $\sigma$-finis. Soit $f : X \times Y \to \overline{R}$ une application $\Fcal \otimes \Gcal$ -mesurable. \\
1) Si $f \geq 0$ alors les applications $\varphi_y : x \mapsto \int_Y f(x,y) d\mu(y)$ et $\psi_x : y \mapsto \int_x f(x,y) d\lambda(y)$ sont mesurables sur,respectivement, $\Fcal$ et $\Gcal$ et $$\int_{X \times Y} f d(\lambda \otimes \mu)=\int_{X} \varphi_y(x) d\lambda(x)=\int_{Y} \psi_x(y) d\mu(y) $$
2) Si $f$ est $\lambda \times \mu$-int�grable alors les applications $f(x,\cdot)$ et $f(\cdot,y)$  sont int�grables pour, respectivement, presque tout $x \in X$ et presque tout $y \in Y$. 
Les applications $\varphi_y$ et $\psi_x$ sont bien d�finis (comme dans le 1)) pour, respectivement, presque tout $y$ et presque $x$ et sont int�grables.
On a, alors, comme dans le cas pr�c�dent, $$\int_{X \times Y} f d(\lambda \otimes \mu)=\int_{X} \varphi_y(x) d\lambda(x)=\int_{Y} \psi_x(y) d\mu(y) $$





\end{Thm}
\vspace{4cm}
\section{Th�or�me de continuit� et de d�rivation sous le signe $\int$}

 Dans la suite, $a$ et $b $ sont deux r�els tels que $a <b $ et $f $ d�signe une application de $
X \times [a,b] $ dans $\R $.
%Pour tout $(_0,t_0)\in X \times   [a,b]$, on note $f (,\cdot) $ et $f (\cdot,t_0) $ les applications respectives $t \in [a,b] \mapsto f (_,t) \in \R $ et $ \in  \mapsto f(,t_0) \in \R $
\begin{Thm}[ de continuit� sous le signe $\int$ ]
Si $f $ v�rifie les trois hypoth�ses suivantes : \\
1) Pour tout $t \in [a,b] $, $f (\cdot,t) \in L^1$ ; \\
2) $\exists g \in L^1, \forall x \in X, \forall t \in [a,b],  |f(x,t)| \leq g(x)$ (hypoth�se de domination) \\
3) Pour tout $ \in X $, l'application  $f (,\cdot) $ est continue sur $[a,b] $ ;\\
alors, l'application  $F: t \in [a,b] \mapsto F(t):=\int_X f(x,t) d\mu(x) \in \R $ est continue sur $[a,b] $.\\
\end{Thm}
\begin{Thm}[de d�rivabilit� sous le signe $\int$ ]
Si $f $ v�rifie les trois hypoth�ses suivantes : \\
1) Pour tout $t \in [a,b] $, $f (\cdot,t) \in L^1$ ; \\
2) Pour tout $(,t)\in X \times [a,b]  $, l'application  $\frac{\partial f}{\partial t}(x,t) $ est d�finie ;\\
3) $\exists g \in L^1, \forall x \in X, \forall t \in [a,b],  \left|\frac{\partial f}{\partial t}(x,t)\right| \leq g(x)$ (hypoth�se de domination). \\
Alors,\\
 1)l'application  $F: t \in [a,b] \mapsto F(t):=\int_X f(x,t) d\mu(x) \in \R $ est d�rivable sur $[a,b] $\\
2) Pour tout $t \in ]a,b [$, $\frac{\partial f}{\partial t}(\cdot,t)\in L^1$ et $F'(t)=\int_X \frac{\partial f}{\partial t}(x,t) d\mu (x) $.
\end{Thm}
\begin{Thm}[G�n�ralisation ]
Soit $k \in \N$ et $f $ qui v�rifie les trois hypoth�ses suivantes : \\
1) Pour tout $t \in [a,b] $, $f (\cdot,t) \in L^1$ ; \\
2) Pour tout $(x,t)\in X \times [a,b]  $, l'application  $\frac{\partial^k f}{\partial t^k}(x,t) $ est d�finie ;\\
3) $\exists g \in L^1, \forall x \in X, \forall t \in [a,b],  \left|\frac{\partial^k f}{\partial t^k}(x,t)\right| \leq g(x)$ (hypoth�se de domination). \\
Alors,\\
 1)l'application  $F: t \in [a,b] \mapsto F(t):=\int_X f(x,t) d\mu(x) \in \R $ est $k$ fois d�rivable sur $[a,b] $\\
2) Pour tout $t \in ]a,b [$, $\frac{\partial^k f}{\partial t^k}(\cdot,t)\in L^1$ et $F^{(k)}(t)=\int_X \frac{\partial^k f}{\partial t^k}(x,t) d\mu (x) $.
\end{Thm}








\ifwhole
 \end{document}
\fi